We will work towards defining a category of palettes for a chosen knot $K$. This will allow us to change rules of colorings as we see fit.

\begin{definition}[palette]
  Let $R$ be a commutative ring with unity, $M$ a finitely generated $R$-module and $\mathcal{C}\subseteq M^3\oplus M^3$ to be a coloring rule confomring to all rules outlined in the previous section. We say that a triplet $(R, M, \mathcal{C})$ is a \buff{palette}.
\end{definition}

Notice that if there is a ring homomorphism $f:R\to S$ then we can consider $M$ as a $S$ module by tensoring with $S$. This allows us to write a morphism between palettes
$$\overline{f}:(R, M, \mathcal{C})\to (S, M_S, \mathcal{C}_S).$$
Similarly, if there is a module homomorphism $g:M\to M'$, then the induced morphism of palettes is
$$\overline{g}:(R, M, \mathcal{C})\to (R, M', \mathcal{C}').$$

\begin{definition}[category of palettes for knot $K$]
  We define $\Col(K)$ to be a \buff{category of palettes} of $K$ with 
  $$\text{Ob}(\Col(K))=\{(R, M, \mathcal{C})\}$$
  being all palettes as described above and for any two palettes
  $$\Hom((R, M, \mathcal{C}), (S, N, \mathcal{K}))$$
  is the set of induced morphisms $\overline{g}$ between modules or $\overline{f}$ between rings, when modules are $M$ and $N=M_S$ respectively.
\end{definition}

It is beneficial to highlight one palette in particular: 
$${(\Z[\Z], \Z[\Z], \{(u, i, (1-t)u+ti)\;:\;u,i\in\Z[\Z]\})},$$
which will be referred to as the \buff{Alexander palette}. We can derive this palette (which was used in \cref{example reduced normal form}) from the Wirtinger presentation of a knot as follows.

Consider a crossing
\begin{center}
  \begin{tikzpicture}
    \draw[<-] (0, 0) node[above] {$o$} --(2, 0) node[above] {$i$};
    \fill[white](1, 0) circle (7pt);
    \draw[->] (1, -1) node[right] {$u$}--(1, 1);
  \end{tikzpicture}
\end{center}
and take some $x$ to be the generator that is used to generate a representation for $K_G^{ab}$. Then, the following is a relation in said group:
$$UxCx(Ux)^{-1}=Ix$$
where $U=ux^{-1}$, $I=ix^{-1}$ and $O=ox^{-1}$. 
We can multiply both sides by $x^{-1}$ to obtain
$$x^{-1}UxCU^{-1}=x^{-1}Ix$$
which is change in $\Z[\Z]$ to
$$
tU+C-U=tI\implies 0=(1-t)U+tI-C
$$
The procedure for the other type of crossing is analogous.

The question that presents itself here is whether using row and column operations one can obtain representation matrix for the Alexander module from coloring matrices.



% Fixing the ring $R$ hints at $(R, 0, 0)$ being a trivial palette and products and coproducts of palettes being defined by their modules:
% $$(R, M, \mathcal{C})\oplus (R, N, \mathcal{K}):=(R, M\oplus N, \mathcal{C}\oplus \mathcal{K}).$$
%
% % \begin{conjecture}
% %   A category of palettes over a fixed ring $R$, $\mathcal{Col}_R(K)$ is Abelian.
% % \end{conjecture}
%
%
%
