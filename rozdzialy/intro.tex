\section{Introduction}

\subsection{Order of an Ideal over PID ring}

PID -> every ideal is generated by one element, every module is an image of a free module, hence it can be expressed as $M\cong R/I_1\oplus...\oplus R/I_n$ for some ideals $I_i$. This allows as to define order of a module as $\ord(M)=\ord(I_1...I_n)$, which is the element that generates the ideal $I_1....I_n$. 

$\ord(M)$ can also be described using equivalence relation $M\sim M_1 + M_2\iff 0\to M_1\to M\to M_2\to 0$ is an exact sequence -> finitely generated abelian groups as $\Z$ modules and vector fields over $\mathfrak{K}$ as $\mathfrak{K}[x]$-modules.

\subsection{The Problem of non-PID rings}

Not every ring is a PID -> we must either find another invariant or make the ring in question a PID. E.g. for $\Z[x, x^{-1}]$ we can tensor it with some field, usually $\Q$ but we might want to try $F_p$ for some prime $p$.

Maybe some example for $\Z[x]$?

\subsection{Short Introduction to Knot Theory?}

Knot - a closed curve immersed in some $3$-dimensional space, or $S^1$ immersed in $S^3$

We will consider only tamed knots? That is knots that can be represented as a sum of a finite amount of straight lines?

Using Mayer-Vietoris sequence we can deduce that $H^1(S^3\setminus K)=\Z$ for any knot $K$. Hence, if we want to find interesting invariants, we must look further. 

Seifert surface of knot $K$ is an orientable surface whose boundary is $K$. We can use it to create an infinite cyclic covering of $S^3\setminus K$ by cutting copies $S^3\setminus K$ along this surface and gluing the $+$ side of Seifert surface of one copy to the $-$ side of the next copy.

$H^1(K^*)$ is more complicated than $H^1(S^3\setminus K)$ and things get interesting if we consider it as a $\Z[\Z]$ (or $\Z[x, x^{-1}]$-module. We can use the fact that $\Pi_1(K^*)^{ab}=H^1(K^*)$ and calculate this module to obtain something called Alexander ideal $I$: $H^1(K^*)\cong \Z[\Z]/I$. If $I$ is a principal ideal, e.g. in the case of trefoil knot of figure eight knot, its generator is called "Alexander polynomial". If this is not the case, we must consider $H^1(K^*; \Q)$ - kohomology module with coefficients in $\Q$, to obtain the Alexander polynomial. In the following paper we will consider what happens if we use $F_p$, a finite field, instead of $\Q$.

The matrix method
