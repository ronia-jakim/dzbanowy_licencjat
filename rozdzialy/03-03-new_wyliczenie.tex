The diagram $D$ has $s$ segments and $x$ crossings.

\subsection*{\centering R1}

\begin{center}
  \begin{tikzpicture}
    \draw[->] (0, 0)--(0, 1);
    \draw (0, 1)--(0, 2);
    \draw[dashed] (0, 1) ellipse (0.8 and 1);

    \begin{knot}[
        consider self intersections,
        ignore endpoint intersections=false,
        flip crossing=1,
        clip width=20pt
      ]
      \strand[->] (-3, 0) to[out=90, in=120] (-2.5, 1.3) to[out=-60, in=60] (-2.5, .7) to[out=-120, in=-90] (-3, 2);
    \end{knot}
    \draw[dashed] (-3, 1) ellipse (0.8 and 1);

    \draw[dashed] (3, 1) ellipse (0.8 and 1);
    \begin{knot}[
        consider self intersections,
        ignore endpoint intersections=false,
        clip width=20pt
        %flip crossing=1
      ]
      \strand[->] (3, 0) to[out=90, in=120] (3.5, 1.3) to[out=-60, in=60] (3.5, .7) to[out=-120, in=-90] (3, 2);
    \end{knot}
    \node at (0, -.5) {$D$};
    \node at (3, -.5) {$D_b'$};
    \node at (-3, -.5) {$D_a'$};

    \node at (-3.3, .6) {$x$};
    \node at (-3.3, 1.4) {$y$};

    \node at (2.7, .6) {$x$};
    \node at (2.7, 1.4) {$y$};

    %\node at (.3, .6) {$w$};

    \draw[<->] (-2, 1)--(-1, 1) node[midway, above] {$R1b$};
    \draw[<->] (2, 1)--(1, 1) node[midway, above] {$R1a$};
  \end{tikzpicture}
\end{center}

The first Reidemeister move allows the following two moves on color checking matrices
$$
\begin{matrix}
  D_a' & & D & & D_b'\\ 
  \begin{bmatrix}
    %&X & Y & \hdots\\ 
    b & a+c  & 0 & \hdots\\ 
    x_1 & y_1 & z_1 \\ 
    \vdots & & & \ddots
  \end{bmatrix} 
       & \overset{D(R1a)}{\sim} &
     \begin{bmatrix}
       x_1 + y_1 & z_1 & \hdots\\ 
       \vdots & & \ddots
     \end{bmatrix} 
       & \overset{D(R1b)}{\sim} &
  \begin{bmatrix}
    %&X & Y & \hdots\\ 
    \beta & \alpha+\gamma  & 0 & \hdots\\ 
    x_1 & y_1 & z_1 \\ 
    \vdots & & & \ddots
  \end{bmatrix} 
\end{matrix}
$$
where $(\forall\;i=1,...,x)\;x_i=0\;\lor y_i=0$.

\begin{proof}[{\bfseries Proof of \cref{warunki na palete}.1.}]
Notice, that if the propagation rule that was outlined at the beginning of this section is to be true, looking closely at the crossing in diagrams above should yield equations
$$0=a+b+c=a+b-1\implies a=1-b$$
$$0=\alpha+\beta+\gamma=\alpha+\beta-1\implies \alpha=1-\beta,$$
as the up and out segments in $D'_a$ and the up and in segments in $D'_b$ must admit coloring with the same element from $M$.
\end{proof}

\subsection*{\centering R2}

\begin{center}
  \begin{tikzpicture}
    \draw[dashed] (0, 0) ellipse (.8 and 1);
    \draw[dashed] (3, 0) ellipse (.8 and 1);

    \begin{knot}[
      % draft mode=crossings, 
      clip width = 4pt,
      flip crossing=1,
      flip crossing=2, 
      % ignore endpoint intersection=false
      ]
      \strand[->] (-75:.8 and 1) to [out=90, in=-90] (0, 0) to[out=90, in=-90] (75:.8 and 1);
      \strand[->] (-105:.8 and 1) to [out=90, in=-90] 
      (.35, 0) to [out=90, in=-90]
      (105:.8 and 1);
    \strand[->] ($(3, 0)+(-75:.8 and 1)$) -- ($(3, 0)+(75:.8 and 1)$);
    \strand[->] ($(3, 0)+(-105:.8 and 1)$) -- ($(3, 0)+(105:.8 and 1)$);
    \end{knot}

    \node at (-.2, -1.5) {$D'$};
    \node at (3, -1.5) {$D$};

    \node at (-.4, 0) {$y$};
    \node at (75:1 and 1.3) {$z$};
    \node at (-75:1 and 1.3) {$x$};
  \end{tikzpicture}
\end{center}

For the second Reidemeister move we will say that $D\phi$ and $D'\phi$ are in relation if they differ by the following matrix move
$$
\begin{matrix}
  D' & & D \\ 
  \begin{bmatrix} 
    b & c & 0 & a & \hdots \\ 
    0 & \beta & \gamma &\alpha \\ 
    x_1 & 0 & z_1 & w_1 \\ 
    \vdots & & & &\ddots
  \end{bmatrix} & \overset{D(R2)}{\sim} &
  \begin{bmatrix} 
       x_1+z_1 & w_1 & \hdots \\ 
       \vdots & & \ddots
     \end{bmatrix}
\end{matrix}
$$
where $(\forall\; i=1,...,x)\;x_i=0\;\lor z_i=0$, in addition to permuting rows and columns and adding linear combination of rows or columns to another row or column.

\begin{proof}[{\bfseries Proof of \cref{warunki na palete}.2.}]
In the case of this Reidemeister move, we would like to be able to color the diagram $D'$ exactly like the diagram $D$ save for the segments contributing to the two additional crossings. This means that segments labeled $z$ and $x$ on the diagram above must admit a coloring with the same element from $M$.

The restrictions stemming from this observation are more easily calculated if homomorphisms $\phi_+$ and $\phi_-$ are made into two matrices, $A_+$ and $A_-$, that the incoming segments (up and in segments) and return the output segments (out and up segments). This is possible because of the propagation rule.
$$
A_+A_-\begin{bmatrix}u\\i\end{bmatrix}=\begin{bmatrix}
  0 & 1 \\ 
  b & a
  \end{bmatrix}\begin{bmatrix}
  \alpha & \beta \\ 
  1 & 0
\end{bmatrix}
\begin{bmatrix}
  u \\ i 
  \end{bmatrix}=\begin{bmatrix}u\\i\end{bmatrix}
$$
Comparing the terms of the matrix $A_+A_-$ with terms of the identity matrix yields:
$$\begin{cases}
  a\beta+\alpha =0 \\ 
  \beta b=1
\end{cases}$$

\end{proof}

\subsection*{\centering R3}

\begin{center}
  \begin{tikzpicture}
    \begin{knot}[
      % draft mode=crossings, 
      clip width=15pt, 
      flip crossing=1, 
      flip crossing = 2, 
      flip crossing=3, 
      flip crossing=4, 
      flip crossing=6, 
      flip crossing=5
      ]
            \strand[->, thick] (3.8, -5)--(6, -2);
      \strand[->, thick] (4.3, -2)--(6.5, -5);
      \strand[<-, thick] (3.8, -4.2)--(6.5, -4.2);

      \strand[->, thick] (9.5, -5)--(11.7, -2);
      \strand[->, thick] (9, -2)--(11.2, -5);
      \strand[<-, thick] (9, -2.8)--(11.7, -2.8);
    \end{knot}

    \draw[dashed] (3.8, -5)--(6.5, -5)--(6.5, -2)--(3.8, -2)--cycle;
    \draw[dashed] (9, -5) rectangle (11.7, -2);
    
    \node at ( 6.5/2 + 3.8/2, -5.5) {$D$};
    \node at (9/2+11.7/2, -5.5) {$D'$};
  \end{tikzpicture}
\end{center}

% {\large\color{purple}DOKOŃCZYĆ - czy tutaj komutujacy diagram cos da? znaczy w sumie to da, ale to jest jakies dzikie zamienianie wspolrzednych}

The last Reidemeister move does not change the size of matrices but only permutes the terms appearing in columns and rows corresponding to the three crossing that are manipulated in the diagram.
$$
\begin{matrix}
  D' & & D \\ 
  \begin{bmatrix}
    \alpha & \gamma & \beta & 0 & 0 & 0 & \hdots \\ 
    0 & 0 & c & b & 0 & a \\ 
    \beta & 0 & 0 & 0 & \gamma & \alpha \\ 
    u_1 & 0 & v_1 & w_1 & x_4 & y_4 \\ 
    \vdots & & & & & & \ddots
  \end{bmatrix} & \overset{D(R3)}{\sim}
     & 
  \begin{bmatrix}
    0 & 0 & \gamma & \beta & \alpha & 0 & \hdots  \\ 
    \beta & 0 & 0 & 0 & \gamma & \alpha \\ 
    0 & c & b & 0 & 0 & a\\ 
    u_4 & 0 & v_4 & w_4 & x_4 & y_4\\ 
    \vdots & & &  & & \ddots
  \end{bmatrix}
\end{matrix}
$$

Let $D(R)$ be the equivalence relation generated by moves $D(R1a)$, $D(R1b)$, $D(R2)$ and $D(R3)$.

\begin{theorem}\label{theorem macierze sa niezmiennikiem}
  For a diagram $D$ colored with an Alexander palette (or its image) the equivalence class of the color checking matrix $D\phi$ under the equivalence relation $D(R)$ is a knot invariant.
  % The equivalence class of a color checking matrix using the Alexander palette, or its image, of a diagram $D$ under relation $D(R)$ generated by matrix relations $D(R1a)$, $D(R1b)$, $D(R2)$ and $D(R3)$ is a knot invariant. Thus we can define $K\phi:=[D\phi]$.
\end{theorem}

\begin{proof}
  A direct result of the definition of the equivalence relation.
\end{proof}

\Cref{theorem macierze sa niezmiennikiem} justifies the following notation {\boldmath$K\phi:=[D\phi]$}.
