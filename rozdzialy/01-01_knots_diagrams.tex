{\large\color{red}KTO CO ROBIŁ}

In mathematical terms, a knot is a particular embedding $S^1\hookrightarrow S^3$. A knot diagram is a {\color{blue}immersive projection} $D:S^1\twoheadrightarrow \R^2$ along a vector such that no three points of the knot lay on this vector \cite{likorish-diagram}.

$S^1$ is an orientable space thus we can choose an orientation for a knot being considered. Then a diagram $D$ is oriented if it is a projection of an oriented $S^1$.

Intuitively, two knots $K_1$ and $K_2$ are equivalent if we can deform one into the other without cutting it and only manipulating it with our hands \cite{murasagi-equivalence}. This translates to equivalence of diagrams, which is generated by a set of moves, called the \buff{Reidemeister moves}. In the case of a diagram without an orientation, three moves are sufficient. When an orientation is imposed on $D$, $4$ diagram moves generate the whole equivalence relation on diagrams \cite{ruchy_zorientowane}.
