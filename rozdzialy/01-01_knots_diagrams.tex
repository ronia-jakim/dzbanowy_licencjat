In mathematical terms, a knot is a particular embedding $S^1\hookrightarrow S^3$. A knot diagram is an {\color{blue}immersive projection} $D:S^1\twoheadrightarrow \R^2$ along a vector such that no three points of the knot lay on this vector \cite{likorish-diagram}.

$S^1$ is an orientable space thus we can choose an orientation for a knot being considered. Then a diagram $D$ is oriented if it is a projection of an oriented $S^1$.

Intuitively, two knots $K_1$ and $K_2$ are equivalent if we can deform one into the other without cutting it and only manipulating it with our hands \cite{murasagi-equivalence}. This translates to equivalence of diagrams, which is generated by a set of moves, called the \buff{Reidemeister moves}. In the case of a diagram without an orientation, three moves are sufficient. When an orientation is imposed on $D$, $4$ diagram moves (pictured in \cref{reidemeister-generating}) generate the whole equivalence relation \cite{ruchy_zorientowane}.

\begin{figure}[h]\centering
  \begin{tikzpicture}
    \begin{knot}[
      consider self intersections, 
      clip width=20pt, 
      %draft mode=crossings,
      flip crossing=4,
      flip crossing=3,
      ]
      \strand[->, thick] (0, 0) to [out=90, in=180] 
      (1, 1.9) to [out=0, in=0, looseness=1.5] 
      (1, 1.1) to[out=180, in=-90] (0, 3);
      \strand[->, thick] (3, 0)--(3, 3);
      \strand[->, thick] (6, 0) to[out=90, in=0] 
      (5, 1.9) to[out=180, in=180, looseness=1.5]
      (5, 1.1) to[out=0, in=-90] (6, 3);
    
    \end{knot}
    \draw[<->] (1.5, 1.5)--(2.5, 1.5) node[midway, above] {$R1a$};
    \draw[<->] (3.5, 1.5)--(4.5, 1.5) node[midway, above] {$R1b$};

  \end{tikzpicture}
  \vspace{1cm}

  \begin{tikzpicture}
    \begin{knot}[
      consider self intersections, 
      clip width=20pt, 
      %draft mode=crossings, 
      flip crossing=2, 
      flip crossing=1, 
      flip crossing=3, 
      flip crossing=4, 
      flip crossing=5,
      flip crossing=6, 
      flip crossing=8, 
      flip crossing=7
      ]
      \strand[->, thick] (1.8, -5)--(1.8, -2);
      \strand[->, thick] (2.2, -5)--(2.2, -2);

      \strand[->, thick] (-.5, -5)--(-.5, -2);
      \strand[->, thick] (-1, -5) to[out=90, in=-90] 
      (0, -3.5) to[out=90, in=-90] 
      (-1, -2);

      \strand[->, thick] (3.8, -5)--(6, -2);
      \strand[->, thick] (4.3, -2)--(6.5, -5);
      \strand[<-, thick] (3.8, -4.2)--(6.5, -4.2);

      \strand[->, thick] (9.5, -5)--(11.7, -2);
      \strand[->, thick] (9, -2)--(11.2, -5);
      \strand[<-, thick] (9, -2.8)--(11.7, -2.8);
    \end{knot}
    
    \draw[<->] (.2, -3.5) -- (1.6, -3.5) node[midway, above] {$R2$};

    \draw[<->] (6.8, -3.5)--(8.7, -3.5) node[midway, above] {$R3$};
  \end{tikzpicture}
  \caption{Generating set of Reidemeister moves in oriented diagrams. \label{reidemeister-generating}}
\end{figure}
