Knowing the resolution of a module allows one to change said module into a matrix or even a sequence of matrices, each containing at least a portion of information about its structure.
We write the following resolution of the Alexander module:
\begin{equation}\label{pierwsza rezolwenta}
  \begin{tikzcd}
    0\arrow[r] & \ker(A_D)\arrow[r] & \Z[\Z]^{n}\arrow[r, "A_D"] & \Z[\Z]^{n-1}\arrow[r, "f"] & K_G^{ab}\arrow[r] & 0
  \end{tikzcd}
\end{equation}
where the $f$ arrow sends every generator $(0,..., 1,...,0)\in\Z[\Z]^{n-1}$ to each of the $(n-1)$ generators of $K_G^{ab}$. In the kernel of this homomorphism are exactly all $n$ relations in $K_G^{ab}$ and they are the image of $A_D$.

\begin{definition}[Alexander matrix]\label{alexander matrix def}
  The matrix $A_D$ in the diagram above is called the \buff{Alexander matrix} of the Alexander module $K_G^{ab}$ using the Wirtinger presentation with the diagram $D$.
\end{definition}

The Alexander matrix in the case of a knot group is not a square matrix. However, striking out any of its columns will give a square matrix whose determinant is nonzero. We will prove this statement promptly after consider the Alexander module as a vector space over the field of fractions of $R=\Z[\Z]$ \cite[Chapter~3]{atiyah}.

In \cref{prop: modul alexandera jest torsyjny} it was shown that the Alexander module is torsion. Let $F=R^{-1}R$. Then, as a vector space $K_G^{ab}\otimes_R F=0$ it is trivial. Hence, the sequence in (\ref{pierwsza rezolwenta}) translates to the following sequence of $F$ modules
\begin{equation}\label{resolution vector spaces}
  \begin{tikzcd}
    0\arrow[r] & \ker(A_D)\otimes_RF \arrow[r] & R^n\otimes_R F \arrow[r, "A_D^V"] & R^{n-1}\otimes_R F \arrow[r] & 0
  \end{tikzcd}
\end{equation}
As there exists an inclusion $R\hookrightarrow F=R^{-1}R$, every matrix with terms in $R$ can be treated as a matrix with terms in $F$. Naturally, $A_D^V=A_D\otimes Id_{F}$ is just matrix $A_D$ (with terms in $R$) with adjoined $1\times 1$ matrix with just identity of $F$. Thus, we can easily translate most properties of $A_D^V$ to properties of $A_D$, i.e. its determinant and surjectivity.%, by forgetting the $R^{-1}R$ factor.

\begin{lemma}\label{alexander matrix has trivial kernel}
  Let $A_D'$ be the Alexander matrix $A_D$ with one of its rows struck out. Then $\det(A_D')\neq 0$.
\end{lemma}

\begin{proof}
  We start by noticing that every crossing contains three segments and so every row of the Alexander matrix has at most three non-zero terms. The relation in Wirtinger presentation generated by crossing 
  \begin{center}
    \begin{tikzpicture}
      \draw(0,0)node[above]{$a$}--(2, 3) node[above] {$b$};
      \fill[white] (1, 1.5) circle (20pt);
      \draw[->](2, 0)--(0, 3) node[above] {$u$};
    \end{tikzpicture}
  \end{center}
  is of form
  $$ubu^{-1}=c.$$
  As described in the previous section, we change the Wirtinger presentation so that only one generator $x$ is send to $1$ by abelianization. If said generator is $u=x$, then in the $\Z[\Z]$ module $K^{ab}$ we see the following relation 
  $$\pm t^n(tB-C)=0,$$
  where $B=bx^{-1}$ and $C=cx^{-1}$. Otherwise, the relation is
  $$\pm t^n[(1-t)U+tB-C]=0,$$
  and the row corresponding to this crossing in the Alexander matrix has exactly three terms.

  In those two cases, the sum of coefficients of $A_D(1)$ in the row corresponding to the crossing is equal to $1$.

  The cases in which $x$ is $b$ or $c$ are symmetrical and without the lose of generality assume that $x=b$. Then the relation is 
  % $$x^{-1}Ux=x^{-1}CxU$$
  $$\pm t^n[(t-1)U-tC]=0.$$
  Notice that the coefficients in row corresponding to this crossing are $0$ and $\pm1$. Thus, the sum is not equal to zero. There are two of such rows as the segment $b$ has to be the "out" and "in" segment of some crossing. In other words, segment $b$ has to have a start and end in some crossings.

  The reasoning above is true for matrix $A_D^V$ from (\ref{resolution vector spaces}). We make the switch to vector space to use the connection between the rank of matrix and the dimension of its image.

  Let $S_i$ be the column of the Alexander matrix corresponding to the segment labeled $i$. The sum $\sum_{i\leq n-1} S_i$ is a vector with two nonzero terms. Take $S_j$ and $S_k$ to be the vectors with those nonzero terms. The only way to cancel out those coordinates is to multiply both $S_j$ and $S_k$ by zero. However, doing this we eliminate two other coordinates with nonzero terms. This yields a sum
  $$\sum_{\substack{i\leq n-1 \\ i\neq j,k}}S_i$$ 
  which still has two nonzero elements. Repeat the reasoning until only one nonzero vector remains or all the vectors are multiplied by $0$.

  We showed that $\{S_i\;:\;i\leq n-1\}$ is a set of linearly independent vectors and thus every minor of $A_D^V(1)$ has nonzero determinant. In particular, $\det(A_D')(1)\neq 0$.
\end{proof}

The \cref{alexander matrix has trivial kernel} implies that image of $A_D^V$ has dimension $(n-1)$. We will use this knowledge later on to construct the resolution of the Alexander module.

\begin{theorem}\label{wyznacznik nie zalezy od diagramu}
  The determinant $\det(A_D')$ up to multiplication by a unit is independent of the choice of the diagram $D$.
\end{theorem}

A proof of this statement using Dehn presentation of knot group rather than the Wirtinger presentation is presented in \cite{alex-oryginal}.

\begin{definition}[Alexander polynomial]
  \hl{Let $p(t)$ be the determinant of any maximal minor of the Alexander matrix $A_D$. Then, we can find $k\in\Z$ such that $t^kp(t)$ is a symmetrical polynomial, meaning that the absolute values of the highest and lowest powers of variables in it are equal} \cite{alex-oryginal}. \hl{The polynomial $t^kp(t)$ is called the }\buff{Alexander polynomial} of a knot $K$.
\end{definition}

The Alexander polynomial is a knot invariant as a consequence of \cref{wyznacznik nie zalezy od diagramu}.

\begin{proposition}
  \hl{Let $G$ be a knot group of $K$ and $F=R^{-1}R$ the field of fractions of ring $R$} \cite{atiyah}. \hl{Then, changing coefficients by applying functor $-\otimes_R F$ to the following exact sequence}
  \begin{center}
    \begin{tikzcd}
      0\arrow[r] & \ker(A_D)\arrow[r] & R^n\arrow[r, "A_D"] & R^{n-1} \arrow[r] & K_G^{ab}\arrow[r] & 0 \\ 
    \end{tikzcd}
  \end{center}
  yields the following exact sequence
  \begin{center}
    \begin{tikzcd}
      0\arrow[r] & F \arrow[r] & F^{n}\arrow[r, "A_D^V"] & F^{n-1}\arrow[r] & K_G^{ab}\otimes_R F=0\arrow[r] & 0
    \end{tikzcd}
  \end{center}
  \hl{where $n$ is the number of crossings of the chosen diagram $D$ of knot $K$.}
  % Let $G$ be a knot group of $K$ and $F=R^{-1}R$ the field of fraction of ring $R$. Then $K_G^{ab}$ always has a resolution
  % \begin{center}
  %   \begin{tikzcd}
  %     0\arrow[r] & M \arrow[r] & R^{n}\arrow[r, "A_D"] & R^{n-1}\arrow[r] & K_G^{ab}\arrow[r] & 0
  %   \end{tikzcd}
  % \end{center}
  % where $n$ is the number of crossings of the chosen diagram $D$ of knot $K$ and $M\otimes_R F\cong F$.
  %
  % \color{blue}tutaj próbowałam ubrać w słowa to, co Pan mówił, że niekoniecznie \begin{tikzcd}0\arrow[r]&R\arrow[r] &R^n\arrow[r] & R^{n-1}\arrow[r]&K_G^{ab}\arrow[r] & 0\end{tikzcd} jest szukanym przeze mnie ciągiem dokładnym, ale jak ostatnie $R$ zastąpię $M$ to powinno działać. Chyba, że powinnam to po prostu wyrazić w ramach $F$-modułów i wtedy nie ma problemu, ale $K_G^{ab}$ znika?
\end{proposition}

\begin{proof} 
  We start by saying that $R^n\otimes_R F\cong (R\otimes_R F)^n$ and $R\otimes_R F\cong F$ \cite[Proposition~2.14]{atiyah}.

   \Cref{alexander matrix has trivial kernel} implies that (\ref{resolution vector spaces}) can be extended into the following exact sequence of vector spaces:
   \begin{center}
     \begin{tikzcd}[column sep=1cm]
       0\arrow[r] & F\arrow[r] & F^n \arrow[r, "A_D^V"]
                  & F^{n-1}\arrow[r] & K_G^{ab}\otimes_RR^{-1}R=0\arrow[r] & 0
     \end{tikzcd}
   \end{center}
   as we proved that $\dim(\im A_D^V)=n-1\implies \dim(\ker A_D^V)=1$.

  The ring of fractions is flat \cite[Chapter~3]{atiyah} at the same time we only consider $R$-modules treated as vector spaces in this proposition. Thus, we have the following exact sequence
  \begin{center}
    \begin{tikzcd}
      0\arrow[r] & M\arrow[r] & R^n\arrow[r, "A_D"]  & R^{n-1}\arrow[r] & K^{ab}_G\arrow[r] & 0
    \end{tikzcd}
  \end{center}
  with $M\otimes_R F\cong F$.
\end{proof}
