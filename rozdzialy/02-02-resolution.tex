The resolution of a module at first glance is in no way a simplification of said module. However, there are multiple ways of distilling simplifications and invariants from the resolution of the Alexander module. {\color{purple}In this section we want to }

We start writing the beginning $K_G^{ab}$ resolution as follows:
\begin{center}
  \begin{tikzcd}
    ...\arrow[r] & \Z[\Z]^{n}\arrow[r, "A_D"] & \Z[\Z]^{n-1}\arrow[r] & K_G^{ab}\arrow[r] & 0
  \end{tikzcd}
\end{center}


\begin{definition}[Alexander matrix]
  The matrix of homomorphism $A_D$ in the diagram above is called the \buff{Alexander matrix} of group $G$ (knot $K$).
\end{definition}

The Alexander matrix in the case of a knot group is not a square matrix. However, striking out any of its rows will give a square matrix whose determinant is nonzero.

\begin{proposition}\label{alexander matrix has trivial kernel}
  Let $A_D'$ be the Alexander matrix $A_D$ with one of its rows struck out. Then $\det(A_D')\neq 0$.
\end{proposition}

\begin{proof}
  % We will show that $\det(A_D')(1)=\pm1$.
  %
  We start by noticing that every crossing contains three segments and so every row of the Alexander matrix has at most three non-zero terms. The relation in Wirtinger presentation generated by crossing 
  \begin{center}
    \begin{tikzpicture}
      \draw(0,0)node[above]{$a$}--(2, 3) node[above] {$b$};
      \fill[white] (1, 1.5) circle (20pt);
      \draw[->](2, 0)--(0, 3) node[above] {$u$};
    \end{tikzpicture}
  \end{center}
  is of form
  $$ubu^{-1}=c.$$
  As described in the previous section, we change the Wirtinger presentation so that only one generator $x$ is send to $1$ by abelianization. If said generator is $u=x$, then in the $\Z[\Z]$ module $K^{ab}$ we see the following relation 
  $$\pm t^n(tB-C)=0,$$
  where $B=bx^{-1}$ and $C=cx^{-1}$. Otherwise, the relation is
  $$\pm t^n[(1-t)U+tB-C]=0,$$
  and the row corresponding to this crossing in the Alexander matrix has exactly three terms.

  In those two cases, the sum of coefficients of $A_D(1)$ in the row corresponding to the crossing is equal to $1$.

  The cases in which $x$ is $b$ or $c$ are symmetrical and without the lose of generality assume that $x=b$. Then the relation is 
  % $$x^{-1}Ux=x^{-1}CxU$$
  $$\pm t^n[(t-1)U-tC]=0.$$
  Notice that the coefficients in row corresponding to this crossing are $0$ and $\pm1$. Thus, the sum is not equal to zero. There are two of such rows as the segment $b$ has to be the "out" and "in" segment of some crossing. In other words, segment $b$ has to have a start and end in some crossings.

  Let $S_i$ be the column of the Alexander matrix corresponding to the segment labeled $i$. The sum $\sum_{i\leq n-1} S_i$ is a vector with two nonzero terms. Take $S_j$ and $S_k$ to be the vectors with those nonzero terms. The only way to cancel out those coordinates is to multiply both $S_j$ and $S_k$ by zero. However, doing this we eliminate two other coordinates with nonzero terms. This yields a sum
  $$\sum_{\substack{i\leq n-1 \\ i\neq j,k}}S_i$$ 
  which still has two nonzero elements. Repeat the reasoning until only one nonzero vector remains or all the vectors are multiplied by $0$.

  We showed that $\{S_i\;:\;i\leq n-1\}$ is a set of linearly independent vectors and thus every minor of $A_D(1)$ has nonzero determinant. In particular, $\det(A_D')(1)\neq 0$.
\end{proof}

The \cref{alexander matrix has trivial kernel} implies that image of $A_D$ has dimension $(n-1)$. We will use this knowledge later on to construct the resolution of the Alexander module.

\begin{theorem}\label{wyznacznik nie zalezy od diagramu}
  The determinant $\det(A_D')$ is independent of the choice of the diagram $D$
\end{theorem}

\begin{proof}
  If $D$ and $D'$ are two diagrams of knot $K$, then they yield equivalent representations of $G=\pi_1(K)$. Thus, the chain of elementary ideals of $A_D$ and $A_{D'}$ are the same according to Fox \cite[Chapter~VII]{fox} from which immediately follows that the determinants of the maximum minors of $A_D$ and $A_{D'}$ are equal.
\end{proof}

\begin{definition}[Alexander polynomial]
  The \buff{Alexander polynomial} of a knot $K$ is the determinant of any maximal minor of the Alexander matrix $A_D$.
\end{definition}

The Alexander polynomial is a knot invariant as a consequence of \cref{wyznacznik nie zalezy od diagramu} and \cref{alexander matrix has trivial kernel}

\begin{proposition}
  Let $G$ be a knot group of $K$. Then it always has a resolution
  \begin{center}
    \begin{tikzcd}
      0\arrow[r] & \Z[\Z]^1 \arrow[r] & \Z[\Z]^{n}\arrow[r] & \Z[\Z]^{n-1}\arrow[r] & K_G^{ab}\arrow[r] & 0
    \end{tikzcd}
  \end{center}
  where $n$ is the number of crossings of the chosen diagram $D$ of knot $K$.
\end{proposition}

\begin{proof}
  Take $R=\Z[\Z]$ and consider its field of fractions $R^{-1}R$. There is an obvious homomorphism $R\to R^{-1}R$ which allows us to work on $A_D$ as if it was a linear map between vector spaces
  \begin{center}
    \begin{tikzcd}[column sep=huge]
      R\otimes_R R^{-1}R \arrow[r, "A_D\otimes_R id_{R^{-1}R}"] & R\otimes_R R^{-1}R
    \end{tikzcd}
  \end{center}
  with $\dim(A_D\otimes_R id_{R^{-1}R})=(n-1)$ as was proven in \cref{alexander matrix has trivial kernel}.

  Thus, the following is an exact sequence of vector spaces
  \begin{center}
    \begin{tikzcd}
      0\arrow[r] & V \arrow[r] & V^n\arrow[r, "A_D'"] & V^{n-1}\arrow[r] & 0
    \end{tikzcd}
  \end{center}
  where $V=R^{-1}R$ and $A_D'=A_D\otimes_R id_{R^{-1}R}$.

  Now consider the following sequence
  \begin{equation}\label{diagramik resolution}
    \begin{tikzcd}
      0\arrow[r] &R\arrow[r] & R^n\arrow[r, "A_D", twoheadrightarrow] & R^{n-1} \arrow[r, twoheadrightarrow] & K_G^{ab}\arrow[r] & 0
    \end{tikzcd}
  \end{equation}
  The only concerning point is the leftmost arrow as it might not be an injection to $\ker A_D$.

  The ring of fractions is flat \cite[Chapter~3]{atiyah}, the module $K_G^{ab}$ is torsion \cref{prop: modul alexandera jest torsyjny} and thus
  $$K_G^{ab}\otimes_R R^{-1}R=0.$$
  Because of that, tensoring the sequence (\ref{diagramik resolution}) by $R^{-1}R$ {\large\color{red}induces an isomorphism between homologies of the sequences above}, wherefore it is exact.
%
%
%
  % Take $R=\Z[\Z]$ and consider its ring of fractions $R^{-1}R$. There is an homomorphism $R\to R^{-1}R$ which allows us to change the coefficients in the resolution of $K_G^{ab}$ as follows:
  % \begin{center}
  %   \begin{tikzcd}
  %     0\arrow[r] & R^a\otimes_R R^{-1}R \arrow[r] & R^{n}\otimes_R R^{-1}R \arrow[r] & R^{n-1}\otimes_R \arrow[dll, rounded corners, to path={--([xshift=2ex]\tikztostart.east) |- ([xshift=-3ex, yshift=-2ex]\tikztostart.south) -| ([xshift=-2ex]\tikztotarget.west) -- (\tikztotarget)
  %     }]\\ 
  %     & K_G^{ab}\otimes_R R^{-1}R \arrow[r] & 0
  %   \end{tikzcd}
  % \end{center}
  % In \cref{prop: modul alexandera jest torsyjny} it was proven that $K_G^{ab}$ is a torsion module and so $K_G^{ab}\otimes_R R^{-1}R=0$. Moreover, in \cref{alexander matrix has trivial kernel} we showed that $A_D$ is a surjection onto $(n-1)$ dimensional module. Combining this knowledge we obtain a short exact sequence of vector spaces, call them $V$, over $R^{-1}R$:
  % \begin{center}
  %   \begin{tikzcd}
  %     0\arrow[r] & V^a\arrow[r] & V^n\arrow[r] & V^{n-1}\arrow[r] & 0
  %   \end{tikzcd}
  % \end{center}
  % which implies that 
  % $$n=n-1+a$$ 
  % and so $a=1$.
\end{proof}


