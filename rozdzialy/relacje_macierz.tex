\subsection{What is a knot coloring}

% \begin{definition}[diagram coloring]
%   Let $x_1,..., x_s\in M$ be labels of arcs in diagram $D$. We will say that $(x_1,...,x_s)\in M^s$ is a \buff{coloring} if for every crossing $\pm$ in $D$ consisting of arcs $u$, $i$, $o$ the following relation is satisfied
%   $$\phi_\pm(u,i,o)=0.$$
% \end{definition}

% Every crossing in the colored diagram $D$ of knot $K$ yields $x$ relations $\phi_\pm(u,i,o)=0$ which we might treat as linear equations of form 
% $$\phi_+(u,i,o)=au+bi+co=0,$$
% $$\phi_-(u,i,o)=\alpha u+ \beta i+ \gamma o=0,$$
% where $u$, $i$ and $o$ are labels assigned to arcs entering some crossing and $a,b,c\in\Hom(M, N)$.
%
% \begin{definition}
%   Matrix $D\phi:M^s\to N^x$ of coefficients taken from relations $\phi_\pm(u,i,o)$ will be called a \buff{color checking matrix}. 
%   %such that every row has only $3$ nonzero terms, corresponding to arcs entering the appropriate crossing. If $\phi_\pm(u,i,o)=a_\pm u+b_\pm i+c_\pm o$ then those terms will be $a_\pm$, $b_\pm$ and $c_\pm$.
% \end{definition}
%
% Notice that $(x_1,..., x_s)$ is a coloring of the diagram $D$ if and only if it is an element of $\ker D\phi$. However, we can choose $\phi$ to have only a trivial kernel, then only one coloring is admissible - assigning a $0$ to every arc of $D$. Thus, to obtain valuable information about the knot $K$ whose diagram is being colored, we must impose the following restrictions on $\phi$.
% %a knot invariant we must impose some restrictions on $\phi$.
