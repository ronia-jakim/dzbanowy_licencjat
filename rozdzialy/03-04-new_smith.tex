The ring $R$ of palette $(R, M, \mathcal{C}_\pm)$ is not necessarily a PID ring, e.g. $\Z[\Z]$ ring of the Alexander palette has ideal $(2, t+1)$ which is not principal. However, usually one can find a PID ring $P$ with homomorphism $R\to P$ which creates a new palette $(P, M\otimes_R P, \mathcal{C}_\pm\otimes_R P)$ derived from $(R, M, \mathcal{C}_\pm)$. Matrices over PID rings have many interesting properties, like having a Smith normal form.

\begin{definition}[Smith normal form]
Take $A\in K\phi$ and consider it as an $n\times n$ matrix with terms in a $P$ by the procedure outlined above. Then there exist a $n\times n$ matrix $S$ and $n\times n$ matrix $T$ such that $SAT$ is of form
$$
\begin{bmatrix}
  a_1 & 0 & 0 & \hdots & 0 & \hdots & 0 \\ 
  0 & a_2 & 0\\ 
  0 & 0 & \ddots & & \vdots & & \vdots\\ 
  \vdots & & & a_r\\ 
  0 & & \hdots & & 0 & \hdots & 0 \\ 
  \vdots & & & & \vdots & & \vdots\\ 
  0 & & \hdots & & 0 & \hdots & 0
\end{bmatrix}
$$
where for every $i$ $a_i|a_{i+1}$. Such a matrix $SAT$ is called the \buff{Smith normal form} of matrix $A$.
\end{definition}
