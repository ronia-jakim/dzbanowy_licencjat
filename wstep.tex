\section*{Introduction}

Let $K$ be a knot embedded in $S^3$ and $D$ its diagram. The fundamental group of the knot complement $\pi_1(S^3-K)$, called the knot group, is a knot invariant. It is a basis for many simpler invariants, such as the Alexander polynomial \cite{alex-oryginal} or the Alexander module (see \cref{alexander module def}) 

On the other hand, one can approach the search for knot invariants from the perspective of diagrams. A coloring is an assignment of elements of a module $M$ to segments of the diagram. 
% If rules are imposed on this assignment, a knot invariant can be obtained by this procedure. 
Assignments that satisfy specific conditions (see \cref{warunki na palete}) yield knot invariants. 
The following thesis is concentrated on finding a connection between the information about a knot $K$ obtained from its group with the information obtained from a diagram coloring.

\Cref{sec1} defines the fundamental concepts of this paper, such as the knot group (see \cref{knot group def}) its metabelianization (see \cref{metab def}) and the Alexander module (see \cref{alexander module def}). Two equivalent definitions for said modules are presented: an algebraic one and a topological one. In a purely algebraic sense, the Alexander module is the abelianized subgroup $[[G, G], [G, G]]$ of the knot group $G$, with $\Z$ action induced by abelianization homomorphism $G\to \Z$, while from a topological point of view it is the first homology module of the infinite cyclic cover (see \cref{inf cyclic cover}). An important result finishing this section is that the Alexander module is a torsion module (see \cref{prop: modul alexandera jest torsyjny}).

\Cref{section2} is dedicated to the Alexander matrix (see \cref{alexander matrix def}) and its properties. We start by showing an algorithm for obtaining the presentation of the Alexander module, which is then used to construct a resolution of said module. The Alexander matrix is defined as the matrix of the homomorphism $\Z[\Z]^n\Z[\Z]^{n-1}$ with cokernel isomorphic to the Alexander module.
%in this sequence for which cokernel is the Alexander module. 
% The Alexander polynomial is defined as the symmetrical determinant of the Alexander matrix and it is shown that this polynomial is never zero for knots.
Over the ring of fractions $F$ \cite[Cahpter~2]{atiyah} of $\Z[\Z]$, this matrix is surjective and creates a particularly interesting short exact sequence of $F$-modules.

In the last \cref{sec3} diagram colorings are discussed, starting with a definition of a palette (see \cref{def paleta}). A set of palettes of particular interest are the Alexander palette and all its images through homomorphisms induced by either a ring or a module homomorphism. 
From diagram coloring a matrix is obtained, called the color checking matrix (see \cref{def:color checking matrix}). 
Properties of palettes that ensure the invariant nature of color checking matrices are outlined and proven. 
%A new matrix is defined, called the reduced Smith normal form (see \cref{reduced normal form def}), and proven that for diagrams colored using the Alexander palette it is a knot invariant. 
Two knots, K11n85 and Kn164, having the same Alexander polynomial, are distinguished using a reduced Smith normal form of color checking matrix (see \cref{reduced normal form def}). The paper ends with a proof that this reduced normal form is a knot invariant for Alexander palettes.

This thesis is a result of cooperation between Julia Walczuk and myself, and is written under supervision of prof. Tadeusz Januszkiewicz.

\vspace{2cm}

\begin{center}
\scalebox{0.9}{
    \begin{tikzpicture}
      \coordinate (a1) at (90:5);
      \coordinate (a2) at (40:3);
      \coordinate (a3) at (-40:3);
      \coordinate (a4) at (-90:5);
      \coordinate (a5) at (-160:3.5);
\coordinate (a6) at (40:1);
      \coordinate (a7) at (20:6);
      \coordinate (a8) at (80:3);
      \coordinate (a9) at (180+25:1);
      \coordinate (a10) at (130:4);
      \coordinate (a11) at (180:5);
      \coordinate (a12) at (-90:3);
      \coordinate (a13) at (-10:6);
      \coordinate (a14) at (110:2);
      \coordinate (a15) at (110:5);

      % \foreach \i in {1,..., 15} \fill (a\i) circle (5pt);

      \begin{knot}[
        consider self intersections, 
        clip width = 20pt, 
        % draft mode=crossings, 
        flip crossing=1, 
        flip crossing=3, 
        flip crossing=8, 
        flip crossing=4, 
        flip crossing=7, 
        flip crossing=10, 
        flip crossing=9
        ]
        \strand[thick] (a1) to[out=0, in=90] 
        (a2) to[out=-90, in=90] 
        (a3) to[out=-90, in=0] 
        (a4) to[out=180, in=-120]
        (a5) to[out=60, in=150]
        (a6) to[out=-30, in=-90] 
        (a7) to[out=90, in=90, looseness=2] 
        (a8) to[out=-90, in=30]
        (a9) to[out=-150, in=-60]
        (a10) to[out=120, in=90] 
        (a11) to[out=-90, in=180]
        (a12) to[out=0, in=-90] 
        (a13) to[out=90, in=20, looseness=1.5]
        (a14) to[out=-160, in=200, looseness=2]
        (a15) to[out=20, in=180] 
        (a1);
      \end{knot}
    \end{tikzpicture}
    % \caption{A diagram for knot $K11n164$.\label{k11n164 diagram}}
}
\end{center}
 
