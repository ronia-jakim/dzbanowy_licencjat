\section*{Abstract}

The knot group $G=\pi_1(K)$ is a starting point for many knot invariants. Alexander matrix is a representation matrix for a subgroup of $G$ and from its determinant, the Alexander polynomial is obtained. Another way of obtaining said polynomial is by considering a coloring matrix which assigns elements of $R$-module $M$ to segments from a diagram $D$ of knot $K$. This approach can be derived from the image of a resolution of Alexander module through the functor $\Hom(-, M)$. Nevertheless, color checking matrices do not instantly yield a knot invariant, however it is possible to define an equivalence relation that identifies matrices stemming from the same knot. This approach is used to distinguish a pair of knots with the same Alexander polynomial. In the end, a way of generalizing the procedure of coloring diagrams is presented in terms of category theory.

\section*{Introduction}

In knot theory distinguishing knots is often a difficult endeavor, usually facilitated by the notion of invariants. An interesting group of knot invariants are polynomial invariants, such as the Alexander polynomial. Another group worth mentioning are knot colorings that can also yield an element of the ring $\Z[\Z]$.

Very often, considering only one invariant is not sufficient, as there are many knots that share its value, i.e. $K11n85$ and $K11n164$ have the same Alexander polynomial. However, a more subtle application of the same method that yields the Alexander polynomial can sometimes distinguish such knots.

In the following paper, connections between the knot group, knot colorings and homology modules of infinite cyclic covering (see \cite{milnor_infinite_cyclic}) will be studied. As an additional exercise, we will show a way of distinguishing already mentioned knots $K11n85$ and $K11n164$.

{\color{red}
The first section of this paper defines the most important terms used in knot theory, as well as introduces the connection between the Alexander module of a knot and the first homology module of an infinite cyclic covering of said knot. 
}


