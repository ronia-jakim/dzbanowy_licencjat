\documentclass[12pt]{article}

\usepackage[T1]{fontenc}
\usepackage[utf8]{inputenc}

\usepackage{microtype}

\usepackage[a4paper, 
  %showframe,
  total={160mm, 250mm},
  top=30mm
  ]{geometry}

\usepackage{multicol}

\usepackage{graphicx}

\usepackage{varwidth}
\usepackage{tikz}

\setlength{\parindent}{0pt}
\begin{document}
\thispagestyle{empty}
\begin{flushright}
  Wrocław, 8 V 2024
\end{flushright}

Weronika Jakimowicz

330006

Julia Walczuk 

332742

Matematyka I stopnia stacjonarnie

Semestr VI
\vspace{1.5cm}

\begin{flushright}
  \begin{varwidth}{7cm}\large\bfseries
    Dziekanat Wydziału Matematyki i Informatyki
  \end{varwidth}
\end{flushright}

\vspace{1.5cm}

\begin{center}\noindent
\begin{varwidth}{9cm}\centering
\textbf{Podanie o zezwolenie na pisanie pracy licencjackiej w języku innym niż wykładowy}
\end{varwidth}
\end{center}
\vspace{1cm}

Zwracamy się z prośbą o umożliwienie nam pisania pracy licencjackiej, zatytułowanej \emph{Kolorowania Foxa i niezmienniki Alexandera} (ang. \emph{Fox knot colorings and Alexander invariants}) w języku angielskim. Promotorem pracy zgodził się być prof. T. Januszkiewicz.
\medskip

Będzie to dla nas możliwość zdobycia umiejętności pisania prac w języku używanym powszechnie w świecie naukowym.
\medskip

Prosimy o pozytywne rozpatrzenie naszej prośby.
\vspace{.5cm}

\begin{flushright}
  Z wyrazami szacunku
  \vspace{.8cm}

  Weronika Jakimowicz
  \vspace{1.8cm}

  Julia Walczuk

  \includegraphics[width=3cm]{podpis.jpg}
\end{flushright}


\end{document}
